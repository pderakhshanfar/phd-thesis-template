\documentclass{propositions}

%% Turn off page numbering for the propositions and make the margins on both
%% sides equal and symmetrical.
\geometry{twoside=false}
\pagestyle{empty}

\RequirePackage{unicode-math}

\setmainfont[Path = fonts/libertinus/, ItalicFont=libertinusserif-italic.otf, BoldFont=libertinusserif-bold.otf, BoldItalicFont=libertinusserif-bolditalic.otf]{libertinusserif-regular.otf}
\setsansfont[Path = fonts/libertinus/, BoldFont=libertinussans-bold.otf, ItalicFont=libertinussans-italic.otf]{libertinussans-regular.otf}
\setmathfont[Path = fonts/libertinus/]{libertinusmath-regular.otf}
\setmonofont[Scale=MatchLowercase, Path=fonts/inconsolata/]{Inconsolata-Regular.ttf}
%% The default style for text is Tahoma (sans-serif).
%\renewcommand*\familydefault{\rmfamily}

\begin{document}

%% Specify the title and author of the thesis. This information will be used on
%% both the English and Dutch side and in the metadata of the final PDF.
\title{Carving Information Sources to Drive\\ Search-based Crash Reproduction and Test Case Generation}
\author{Pouria}{Derakhshanfar}

\begin{center}

{\Large\titlefont\bfseries Propositions}

\medskip

accompanying the dissertation

\medskip

%% Print the title.
{\makeatletter
\titlestyle\bfseries\large\@title
\makeatother}

%% Print the optional subtitle.
{\makeatletter
\ifx\@subtitle\undefined\else
    \titlefont\titleshape\@subtitle
\fi
\makeatother}

\medskip

by

\medskip

%% Print the full name of the author.
\makeatletter
{\large\titlefont\bfseries\@firstname\ {\titleshape\@lastname}}
\makeatother

\end{center}

\bigskip

\begin{enumerate}
\item Search-based test generation approaches cannot find all faults, but they can find many undetected faults and reproduce reported crashes. [This thesis]
\item The search-based white-box test generation community should put more effort into test levels beyond unit testing. There are many opportunities such as testing interactions between classes and modules. [This thesis]
%  to generate tests covering new scenarios,  
\item Companies joining publicly funded projects (\textit{e.g.,~} EU projects) should make their codes available for academic partners and let them publish about them.
\item Submissions studying open-source subjects that do not have a replication package should be desk-rejected.
\item There should be a structured way to report inadequate reviews, so that reviewers can be addressed, be given feedback on their review performance, and in some cases, be excluded from future reviews for that venue.
\item In the COVID-19 crisis,  given many international employees, universities should pay more attention to the employees living alone.
\item Attending costs of big conferences should be reduced and more transparent.
%  Also, it should be possible to make a successful career without attending these costly conferences.
\item Universities should provide a well-advertized and easy-access infrastructure to regularly checking the mental health of Ph.D. students and PostDocs. 
% Also, they should encourage employees to use this service and educate them in order to normalize it.
\item The Ph.D. study should not turn it into a paper publication competition with fellow Ph.D. students, but rather help them in their endeavors. 
% This healthy communication leads to higher scientific impacts in the long term.
\item Providing a written reply to rebuttals submitted by authors should become mandatory in all venues. This reply helps the author to improve the paper.
\end{enumerate}

\bigskip

%% Apart from the name and title of the supervisor, the following text is
%% dictated by the promotieregelement.
\begin{center}

These propositions are regarded as opposable and defendable, and have been approved as such by the
promotors prof.\ dr.\ A.\ van Deursen, dr.\ A.\ Zaidman, and dr.\ A.\ Panichella, and the daily supervisor dr.\ X.\ Devroey.
\end{center}

%% \clearpage
%% {\selectlanguage{dutch}

%% \begin{center}

%% {\Large\titlefont\bfseries Stellingen}

%% \bigskip

%% behorende bij het proefschrift

%% \bigskip

%% %% Print the title.
%% {\makeatletter
%% \titlestyle\bfseries\large\@title
%% \makeatother}

%% %% Print the optional subtitle.
%% {\makeatletter
%% \ifx\@subtitle\undefined\else
%%     \titlefont\titleshape\@subtitle
%% \fi
%% \makeatother}

%% \bigskip

%% door

%% \bigskip

%% %% Print the full name of the author.
%% \makeatletter
%% {\large\titlefont\bfseries\@firstname\ {\titleshape\@lastname}}
%% \makeatother

%% \end{center}

%% \bigskip
%% \bigskip

%% \begin{enumerate}

%% \item Stelling 1.
%% \item Stelling 2.
%% \item Stelling 3.
%% \item Stelling 4.
%% \item Stelling 5.
%% \item Stelling 6.
%% \item Stelling 7.
%% \item Stelling 8.
%% \item Stelling 9.
%% \item Stelling 10.

%% \end{enumerate}

%% \bigskip
%% \bigskip

%% %% Apart from the name and title of the supervisor, the following text is
%% %% dictated by the promotieregelement.
%% \begin{center}
%% Deze stellingen worden opponeerbaar en verdedigbaar geacht en zijn als zodanig goedgekeurd door de promotoren prof.\ dr.\ A.\ van Deursen and dr.\ A.\ Zaidman.
%% \end{center}

%% }

\end{document}

