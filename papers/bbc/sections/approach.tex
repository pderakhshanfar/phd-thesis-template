
\section{Basic Block Coverage}
\label{section:bbc:approach}

\subsection{Motivating Example} 

During the search process, the fitness of a test case is evaluated using a fitness function, either \WS or \integ. Since the search-based crash reproduction techniques model this task to a minimization problem, the generated test cases with lower fitness values have a higher chance of being selected and evolved to generate the next generation.
One of the main components of these fitness functions is the coverage of specific statements pointed by the given stack trace. The distance of the test case from the target statement is calculated using the approach level and branch distance heuristics. As we have discussed in Section~\ref{section:background:distance}, the approach level and branch distance cannot guide the search process if the execution stops because of implicit branches in the middle of basic blocks (\eg a thrown \texttt{NullPointerException} during the execution of a basic block). As a consequence, these fitness functions may return the same fitness value for two tests, although the tests do not cover the same statements in the block of code where the implicit branching happens.

For instance, assume that the search process for reproducing the crash in Figure \ref{lst:stacktrace} generates two test cases T$_1$ and T$_2$. The first step for these test cases is to cover frame 3 in the stack trace (line \texttt{413} in BaseClass). However, T$_1$ stops the execution at line \texttt{404} due to a \texttt{NullPointerException} thrown in method \texttt{getName}, and T$_2$ throws a \texttt{NullPoint\-erException} at line \texttt{405} because it passes a null value input argument to \texttt{map}.
Even though T$_2$  covers more lines, the combination of approach level and branch distance returns the same fitness value for both of these test cases: approach level is 2 (nodes \texttt{407-408} and \texttt{410}) and branch distance is measured according to the last predicate. This is because these two heuristics assume that each basic block is atomic, and by covering line \texttt{404}, it means that lines \texttt{405} and \texttt{406} are covered, as well.

\subsection{Secondary Objective}

The goal of the Basic Block Coverage (\bbc) secondary objective is to prioritize the test cases with the same fitness value according to their coverage within the basic blocks between the closest covered control dependency and the target statement. At each iteration of the search algorithm, test cases with the same fitness value are compared with each other using \bbc. Algorithm \ref{list:bbc} presents the pseudo-code of the \bbc calculation. Inputs of this algorithm are two test cases T$_1$ and T$_2$, which both have the same fitness value (calculated either using \WS or \integ), as well as line number and method name of the target statement. This algorithm compares the coverage of basic blocks on the path between the entry point of the CFG of the given method and the basic block that contains the target statement (called \textit{effective blocks} hereafter) achieved by  T$_1$ and T$_2$. If \bbc determines there is no preference between these two test cases, it returns 0. Also, it returns a value $<0$ if T$_1$ has higher coverage compared to T$_2$, and vice versa. A higher absolute value of the returned integer indicates a bigger distance between the given test cases.
\begin{algorithm}[H]
    \label{list:bbc}
    \SetAlgoLined
    \KwIn{test T$_1$, test T$_2$, String method, int line}
    \KwOut{int}
    FCB$_1$ $\gets$  fullyCoveredBlocks(T$_1$,method,line)\;
    FCB$_2$ $\gets$  fullyCoveredBlocks(T$_2$,method,line)\;
    SCB$_1$ $\gets$  semiCoveredBlocks(T$_1$,method,line)\;
    SCB$_2$ $\gets$  semiCoveredBlocks(T$_2$,method,line)\;
    
    \uIf{FCB$_1$ $\subset$ FCB$_2$ $\wedge$ SCB$_1$ $\subset$ SCB$_2$) $\vee$ (FCB$_2$ $\subset$ FCB$_1$ $\wedge$ SCB$_2$ $\subset$ SCB$_1$n}{
        return size(FCB$_2$ $\cup$ SCB$_2$) - size(FCB$_1$ $\cup$ SCB$_1$)\;
    }
    \uElseIf{FCB$_1$ = FCB$_2$ $\wedge$ SCB$_1$ = SCB$_2$}{
        closestBlock $\gets$ closestSemiCoveredBlocks(SCB$_1$, method, line)\;
        coveredLines1 $\gets$ getCoveredLines(T$_1$,closestBlock)\;
        coveredLines2 $\gets$ getCoveredLines(T$_2$,closestBlock)\;
        return size(coveredLines2) - size(coveredLines1)\;    
    }
    \Else{
        return 0\;
    }
        
     \caption{\bbc secondary objective computation algorithm}
\end{algorithm}



In the first step, \bbc detects the effective blocks fully covered by each given test case (\ie the test covers all of the statements in the block) and saves them in two sets called \texttt{FCB$_1$} and \texttt{FCB$_2$} (lines 1 and 2 in Algorithm \ref{list:bbc}). Then, it detects the effective blocks semi-covered by each test case (\ie blocks where the test covers the first line but not the last line) and stores them in \texttt{SCB$_1$} and \texttt{SCB$_2$} (lines 3 and~4). 
The semi-covered blocks indicate the presence of implicit branches. Next, \bbc checks if both fully and semi-covered blocks of one of the tests are subsets of the blocks covered by the other test (line 5). In this case, the test case that covers the most basic blocks is the winner. Hence, \bbc returns the number of blocks only covered by the winner test case (line 6). If \bbc determines T$_2$ wins over T$_1$, the returned value will be positive, and vice versa. 

If none of the test cases subsumes the coverage of the other one, \bbc checks if the blocks covered by T$_1$ and T$_2$ are identical (line 7). If this is the case, \bbc checks if one of the tests has a higher line coverage for the semi-covered blocks closest to the target statement (lines 8 to 11). If this is the case, \bbc will return the number of lines in this block covered only by the winning test case. If the lines covered are the same for T$_1$ and T$_2$ (\ie \texttt{coveredLines1} and \texttt{coveredLines2} have the same size), there is no difference between these two test cases and \bbc returns value 0 (line 11).
Finally, if each of the given tests has a unique covered block in the given method (\ie the tests cover different paths in the method), \bbc cannot determine the winner and returns 0 (lines 12 and 13) because we do not know which path leads to the crash reproduction.

\textbf{Example. } 
%
When giving two tests with the same fitness value (calculated by the primary objective) T$_1$ and T$_2$ from our motivation example to \bbc with target method \texttt{fromMap} and line number 413 (according to the frame 3 of Figure \ref{lst:stacktrace}), this algorithm compares their fully and semi-covered blocks with each other. 
In this example both T$_1$ and T$_2$ cover the same basic blocks: the fully covered block is \texttt{403} and the semi-covered block is \texttt{404-406}. So, \bbc checks the number of lines covered by T$_1$ and T$_2$ in block \texttt{404-406}. Since T$_1$ stopped its execution at line 404, the number of lines covered by this test is 1. In contrast, T$_2$ managed to execute two lines (404 and 405). Hence, \bbc returns $size(coveredLines2) - size(coveredLines1) = 1$. The positive return value indicates that T$_2$ is closer to the target statement and therefore, it should have higher chance to be selected for the next generation.

\textbf{Branchless Methods. } 
%
\bbc can also be helpful for branchless methods. Since there are no control dependent nodes in branchless methods, approach level and branch distance cannot guide the search process in these cases. For instance, methods from frames 1 and 2 in Figure \ref{lst:stacktrace} are branchless. So, we expect that \bbc can help the current heuristics to guide the search process toward covering the most in-depth statement.