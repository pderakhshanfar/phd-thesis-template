\section{Conclusion and Future work} 
\label{sec:bbc:confut}

Approach level and branch distance are two well-known heuristics, widely used by search-based test generation approaches to guide the search process towards covering target statements and branches. These heuristics measure the distance of a generated tests from covering the target using the coverage of control dependencies. However, these two heuristics do not consider implicit branches. For instance, if a test throws an exception during the execution of a non-branch statement, approach level and branch distance cannot guide the search process to tackle this exception. In this paper, we introduced a secondary objective called \bbc to address this issue. To assess \bbc, we used it for search-based crash reproduction due to the high chance of implicit branch occurrence and the limited number of basic blocks that should be covered.
Our results show that \bbc helps \integ and \WS to reproduce 6 and 1 new crashes, respectively.
Also, \bbc significantly improves the efficiency in 26.6\% and 13.7\% of the crashes using \integ and \WS,~respectively.

In our future work, we will investigate the application of \bbc for other search-based test generation techniques (such as unit and  integration).
\vspace{-1em}