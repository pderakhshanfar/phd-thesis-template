
\subsection{Test Generation Cost}

One of the challenges in automated class integration testing is detecting the integration points between classes in a SUT. The number of code elements (e.g., branches) that are related to the integration points increases with the complexity of the involved classes. Finding and testing a high number of integration code targets increases the time budget that we need for generating integration-level testing. 

With CBC, the number of coupled branches to exercise is upper bounded to the cartesian product between the branches in the caller $R$ and the callee $E$. Let $B_R$ be the set of branches in $R$ and $B_E$ the set of branches in $E$, the maximum number of coupled branches $CB_{R,E}$ is $B_R \times B_E$. In practice, the size of $CB_{R,E}$ is much smaller than the upper bound as the target branches in the caller and callee are subsets of $R$ and $E$, respectively. Besides, CBC is defined for pairs of classes and not for multiple classes together. This substantially reduces the number of targets we would incur when considering more than two classes at the same time.

%Our empirical evaluation shows that executing \integration with a small budget (five minutes) complement the unit-level test case generation. Such overhead is comparable with the running time needed for generating unit tests for one single class. Therefore, software testers can create suites that combine the strengths of generated unit-testing with the effectiveness of the integration test suites produced by \cling. 

\subsection{Effectiveness}

To answer \textbf{RQ2}, we analyzed the set of mutants that are killed by \cling (integration tests), but not by the two unit-test suites generated by \evosuite for the caller and callee separately (boxes labeled with $T_{E+R}$ in Figure~\ref{fig:mutation:diff:boxplot}). Note the test suite $T_\cling$ was generated by \cling using a search budget of five minutes. Similarly, the unit-level suites $T_{E}$ and $T_{R}$ were generated with a search budget of five minutes for each class separately. Therefore, the total search budget for unit test generation ($T_{E+R}$) is 10 minutes. Despite the larger search budget spent on unit testing, there are still mutants and faults detected only by \cling and in less time.

Note that \cling is not an alternative tool to unit testing tools like \evosuite. In fact, integration test suites do not subsume unit-level suites as the two types of suites focus on different aspects of the SUT. Our results (\textbf{RQ2}) confirm that integration and unit testing are complementary. Indeed, some mutants can be killed exclusively by unit-test suites: the overall mutation scores for the unit tests $E$, and $R$ are larger than the overall mutation scores of \cling. This higher mutation score is expected due to the larger unit-level branch coverage achieved by the unit tests (coverage is a necessity but not a sufficient condition to kill mutant). 

Instead, \cling focuses on a subset of the branches in the units (caller and callee), but exercises the integration between them more extensively. In other words, the search is less broad (few branches), but more in-depth (the same branches are covered multiple times within different pairs of coupled branches). This more in-depth search allows killing mutants that could not be detected by satisfying unit-level criteria. %In branch coverage, a test suite is adequate if it covers all branches in the code independently of the quality of the test inputs and output. With CBC, the test input data must allow the execution of a branch in the caller
Our results further indicate that it also allows us finding bugs that are not detectable by unit tests.

%In summary, generate unit and integration tests do not subsume or dominate one another. Instead, they complement one another and they are both important to achieve higher test effectiveness.

%On the contrary with line and branch coverage, \emph{\evosuite-5min + \integration-5min} complements \emph{\evosuite-10min} in mutation coverage. In about $ 33 \% $ of class pairs, selected for this study, running \integration and \evosuite for 5 minutes reaches a significantly higher mutation coverage, compared to running \evosuite for 10 minutes. The interesting point is that despite the lower line and branch coverage of target class obtained by \emph{\evosuite-5min + \integration-5min}, it kills more mutants in this class. For instance, the most significant improvement achieved by \emph{\evosuite-5min + \integration-5min} for mutation coverage is in \emph{mockito} project, while, as demonstrated by Figure \ref{fig:coverages}, the line and branch coverage of \emph{\evosuite-5min + \integration-5min} is lower than \emph{\evosuite-10min} in this project.

%The other interesting point about the achieved mutation coverage  by \emph{\evosuite-5min + \integration-5min} is that in 20 class pairs we observe more than $10 \% $ improvement in mutation coverage (maximum is $ 37 \%$ when it tests integration between \texttt{lang3.time.FastDateFormat} and \texttt{lang3.time.FastDateFormat\$TextField} in \emph{lang}). However, there are only two class pairs in which \emph{\evosuite-5min + \integration-5min} reduces the mutation coverage more than $10 \%$ (maximum is $15 \%$ when it tests integration between \texttt{rhino.head.tools.shell.Global} and \texttt{rhino.head.ScriptRuntime} in \emph{closure})


%Moreover, we should note that we selected the most coupled and complex classes for this study. Hence, giving only five minutes to \evosuite was not enough. If we run each of the tools with a higher budget, the negative impact of running \emph{\evosuite-5min + \integration-5min} will reduce.

%In addition to mutation coverage, this study confirms that \integration can capture some extra unexpected crashes compared to \evosuite. In the stack traces of all of these unexpected crashes, we can observe the integration between multiple classes. Since \evosuite focuses on testing only one class, it cannot find these crashes.

\subsection{Applicability}

\cling considers pairs of classes and exercises the integration between them. We did not propose any procedure for selecting pairs of classes to give in input to \integration. However, \cling can be applied to any pair of classes in which at least one of the classes calls the other one. Besides, our approach can be further extended by incorporating integration test ordering approaches and selecting the classes to integrate with a given ordering.

In this paper, we consider only the integration call type of integration between classes, although other types of integration exist between classes~\cite{Offutt2000b} (i.e., integration through external data). 
However, our results are very encouraging because they show \textit{how integration-level tests based on CBC coverage complement unit-level tests generated with \evosuite in terms of test effectiveness}. Further research is needed to incorporate other types of integrations in \cling. This is part of our future agenda.
