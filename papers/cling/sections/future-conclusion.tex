Previous studies have introduced many automated unit and system-level testing approaches for helping developers to test their software projects. However, there is no approach to automate the process of testing the integration between classes, even though this type of testing is one of the fundamental and labor-intensive tasks in testing. Therefore, in this chapter we have introduce a testing criterion for integration testing, called the \textit{Coupled Branches Criterion} (CBC). Furthermore, we have presented an evolutionary-based class integration testing approach called \integration that uses the CBC criterion to generate these kinds of tests with a low budget.

In our investigation of 140 branch pairs that we collected from 5 open source Java projects, we found that \integration has reached an average CBC score of 49\% across all classes, while for some classes we reached 90\% coverage. More tangibly, if we consider mutation coverage and compare automatically generated unit tests with automatically generated integration tests using the \integration approach, we find that our approach allows to kill 7.7\% of mutants per class that cannot be killed by unit tests generated with EvoSuite, despite a larger search budget. Finally, we identified 29 faults causing system crashes that could be evidenced only by the generated class-integration tests. 

The results indicate a clear potential application perspective, more so because our approach can be incorporated into any integration testing practice. Additionally, \integration can be applied in conjunction with with other automated unit and system-level test generation approaches in a complementary way.

From a research perspective, our study shows that \integration is not an alternative for unit testing. However, it can be used for complementing unit testing for reaching higher mutation coverage and capturing additional crashes which materialize during the integration of classes. These improvements of \integration are achieved by the key idea of using existing usages of classes in calling classes in the test generation process. 

For now, \integration only tests the call-coupling between classes. In our future work, we will extend this approach to other types of coupling between classes (\eg Parameter coupling, Shared data coupling, and External device coupling). Moreover, this study indicates that despite the effectiveness of \integration in complementing unit tests, lots of objectives (coupled branches) remain uncovered during our search process. Hence, in future studies, we will (i) detect the infeasible branches and remove them from the search objectives; and (ii) find and tackle the challenges during this search process to cover more integrations between classes. Also, this study mostly focuses on examining the results of this approach on coupled branches coverage, mutation coverage, and detected faults. In future studies, it would be interesting to create a benchmark dedicated to class integration bugs and evaluate our approach using this benchmark.