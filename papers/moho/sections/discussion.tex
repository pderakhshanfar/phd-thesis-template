\section{Discussion}
\label{sec:discussion}

\subsection{Effectiveness and applicability}
Generally, \decomposition reproduces some crashes that cannot be reproduced by \SGGA due to its improved exploration ability, resulting from the multi-objectivization of the crash distance. However, since the decomposed objectives in this approach depend on one another (\eg the stack trace similarity is not helpful if the generated test does not throw the given type of exception), they may misguide the search process in various cases. For instance, as we saw in Section \ref{sec:results:rq2}, \SGGA reproduces six crashes that are not reproducible by \decomposition.

In contrast, \moho has three \textit{conflicting} search objectives. From the theory~\cite{jensen2004helper}, the objective function must be conflicting to increase the overall exploration ability. Our results confirm the theory: the chance of the search process getting trapped in a local optimum is lower by using \moho objectives compared to the ones used in \decomposition. As we observed in Section \ref{sec:results:rq2}, after 1 minute of search, \moho reproduces 8 and 7 crashes more than \SGGA and \decomposition, respectively. Also, it continues outperforming with larger search budgets (2, 3, 4, and 5 minutes) until the end of the search process. It reproduces 5 and 6 crashes more than \SGGA and \decomposition, respectively, while it cannot reproduce only two crashes, reproduced by the other algorithms.

Note that reproducing each crash needs a particular test case which drives the software under test to a particular state, and then, it calls a method with proper input variables. To achieve this goal, each crash reproducing test case needs to create multiple complex objects. Hence, reproducing five new crashes (4\% of crashes available in our benchmark) is a significant improvement for \moho.

\subsection{Factors in the benchmark crashes that impact the Success of \moho}
\label{sec:discussion:factors}

There are multiple factors/characteristics of the crashes in our benchmark that might impact the performance of our approach positively. We identify the following relevant factors: (1) the type of the exception (e.g., null pointer exception), (2) the size the stack frames, (3) the number of classes involved in the crashes, (4) the number of methods of the deepest class in the crash stack. To verify whether these factors influence the performance of our algorithm, we used the two-way permutation test~\cite{pesarin2010permutation}. The permutation test is a well-established non-parametric to assess the significance of factor interactions in multi-factorial analysis of variance (non-parametric ANOVA). We use a significance level $alpha$=0.05 and a very large number of iterations (1,000,000) to ensure the stability of the results over multiple executions of the procedure~\cite{pesarin2010permutation}. 

For the sake of our analysis, we considered the difference in crash reproduction rate between \moho and the baselines as the dependent variable, while the co-factors are our independent variables. According to the permutation test, the type of exception ($p$-value=0.006) and the number of crash stack frames ($p$-value=0.001) significantly impact the performance of \moho compared to Single Objective Search. We can also observe similar results when considering the improvements of \moho against \decomposition: $p$-values=$<10^{-12}$ for both exception type and the number of frames). In other words, there are certain types of exceptions and stack trace sizes for which \moho is statistically better than the state-of-the-art approaches.

From a deeper analysis, we observe that for \texttt{NullPointerExcep}-\texttt{tion} and \texttt{org.joda.time.IllegalFieldValueException}, \moho achieves a higher reproduction ratio than Single Objective Search when the stack traces contain up to three frames for \texttt{NPE} (+22\% in reproduction rate) and up to five frames for \texttt{IllegalFieldValueEx}-\texttt{ception} (+50\% in reproduction rate). Instead, for stack traces with more frames, the differences in reproduction ratio are negligible ($\pm 1\%$ on average) or negative (-10\% in reproduction ratio). Besides, \moho achieves better reproduction ratios for the following exceptions independently of the stack size: \texttt{XWikiExceptions} (+23\% on average), \texttt{UnsupportedOperationException} (+6\% on average), \texttt{MathRuntimeException} (+14\% on average).

Finally, \moho outperforms \decomposition when reproducing \texttt{NullPoin}-\texttt{terException} with 1-3 frames (+8\% on average), \texttt{ClassCastExcep}-\texttt{tion} (+8\% on average), \texttt{StringOutOfBoundsException} (+18\% with more than 2 frames, on average), 
\texttt{IllegalFieldException} (+8\% on average),
\texttt{UnsupportedOperationException} (+23\% on average),
\texttt{MockitoException} (+83\% for short traces, on average), and \texttt{MissingMethodInvocation} (+80\% on average).

\subsection{Crash reproduction cost}
In this study, we observed that since \moho increases the diversity of the generated test cases, it can dramatically improve the efficiency of crash reproduction. This algorithm significantly improved the speed of the search process in more than 36\% of crashes compared to \SGGA and \decomposition. In cases in which \moho had a significant impact, it improves the crash reproduction speed by more than 47\%.

Our prior study (Chapter 2) suggested 5 minutes as the search budget because the search process cannot reproduce more after 5 minutes. However, we observed that despite the high efficiency of \moho, this algorithm continues to reproduce more crashes in the second half of the time budget. Section \ref{sec:results:rq2} shows that \moho keeps increasing the crash reproduction ratio even in the last minutes of the search process, while the previous multi-objectivization approach (\decomposition) changes only slightly after the first 2 minutes of crash reproduction. Hence, increasing the search budget for \moho can lead to a higher crash reproduction ratio.


\subsection{Extendability}

The improvement achieved by the proposed helper-objectives shows the impact of suitable objectives on increasing the diversity of the generated test cases and result in improving the effectiveness and efficiency of the crash reproduction search process. Hence, we hypothesize that this approach can be extended by adding new relevant helper-objectives.