Writing a test case reproducing a reported software crash is a common practice to identify the root cause of an anomaly in the software under test. However, this task is usually labor-intensive and time-taking. Hence,
evolutionary intelligence approaches have been successfully applied to assist developers during debugging by generating a test case reproducing reported crashes.  These approaches use a single fitness function called \CrashFunction to guide the search process toward reproducing a target crash. Despite the reported achievements, these approaches do not always successfully reproduce some crashes due to a lack of test diversity (premature convergence). In this study, we introduce a new approach, called \moho, that addresses this issue via multi-objectivization. In particular, we introduce two new Helper-Objectives for crash reproduction, namely \textit{test length} (to minimize) and \textit{method sequence diversity} (to maximize), in addition to \CrashFunction.
We assessed \moho using five multi-objective evolutionary algorithms (NSGA-II, SPEA2, PESA-II, MOEA/D, FEMO) on 124 non-trivial crashes stemming from open-source projects. Our results indicate that SPEA2 is the best-performing multi-objective algorithm for \moho.
We evaluated this best-performing algorithm for \moho against the state-of-the-art: single-objective approach (\SGGA) and decomposition-based multi-objectivization approach (\decomposition). Our results show that \moho reproduces five crashes that cannot be reproduced by the current state-of-the-art. Besides, \moho improves the effectiveness (+10\% and +8\% in reproduction ratio) and the efficiency in 34.6\% and 36\% of crashes (i.e., significantly lower running time) compared to  \SGGA and \decomposition, respectively. For some crashes, the improvements are very large, being up to +93.3\% for reproduction ratio and -92\% for the required running time. 