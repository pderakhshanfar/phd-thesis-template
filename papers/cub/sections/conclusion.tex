

%%%%%%%%%%%%%%%%%%%%%%%%%%%%%%%%%%%%%%%%%%
\section{Conclusion and future work}
\label{sec:cub:conclusion}
%%%%%%%%%%%%%%%%%%%%%%%%%%%%%%%%%%%%%%%%%%

In this paper, we introduced the \emph{commonality score}, denoting how close an execution path is from common or uncommon executions of the software in production, and the \com and \ucom secondary objectives for search-based unit test generation. 
We implemented our approach in \evosuite and evaluated it on \jabref using execution data from real usages of the application. 
Our results are mixed. The \com secondary objective leads to an increase of the commonality score, and the \ucom secondary objective leads to a decrease of the score, compared to the \df secondary objective (\textbf{RQ.1}).
However, results also show that if the commonality score can have a positive impact on the structural coverage (\textbf{RQ.2}) and mutation score (\textbf{RQ.3}) of the generated test suites, it can also be detrimental in some cases. 
Future research includes a replication of our evaluation on different applications and using different algorithms (\eg \textit{DynaMOSA}) to gain a deeper understanding of when to apply \com and \ucom secondary objectives; the exploration and assessment of different definitions of commonality; and an assessment of the generated tests regarding their usefulness for debugging. 


