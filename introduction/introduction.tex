\chapter{Introduction}
\label{introduction}

\begin{abstract}
Sample Abstract. 
\end{abstract}

\blfootnote{This chapter is partly based on \faFileTextO~\emph{}~\cite{}.
 }


\newpage

\dropcap{T}his is a introductory page: use ICST

\section{Background \& Context}
\subsection{Search-based Sofrware Test Generation}

\subsection{Automated Crash Reproduction}

\section{Research Goals \& Questions}
Research questions:


\textbf{RQ$_1$: } \textit{What are the challenges in search-based crash reproduction?}

\textbf{RQ$_2$: } \textit{Based on the identified challenges, how can we leverage the existing knowledge, carved from information sources, to steer the crash reproduction search process?}

\textbf{RQ$_3$: } \textit{How can we utilize the existing knowledge, carved from information sources, for designing search-based test generation approaches for other criteria?}


\section{Research Outline}

Chapters:

\textbf{Chapter 2: } JCrashPack paper (adressing $RQ_1$)


\textbf{Chapter 6: } Model seeding (adressing $RQ_2$)

\textbf{Chapter ?: } Multi-objectivization paper (adressing $RQ_2$)

\textbf{Chapter 4: } MOHO paper (adressing $RQ_2$)

\textbf{Chapter 5: } Basic Block Coverage paper (adressing $RQ_2$)

\textbf{Chapter 3: } CLING paper (adressing $RQ_3$)

\textbf{Chapter ?: } Common behavior (adressing $RQ_3$)

\textbf{Chapter 7: } Conclusion


\section{Research Methodology}

Design science \cite{Hevner2004}

\section{Open Science}

% In this thesis \cite{Derakhshanfar2020}, you can reference pictures~\Cref{fig:devmodel} using Cleverref and circles \circled{5}.

% \begin{figure}[htb]
% 	\centering
% 	\includegraphics[width=0.65\columnwidth]{development_model_without_papers}
% 	\caption{The stages of the FDD model and their relationship to other
%           Software Engineering concepts.}
% 	\label{fig:devmodel}
% \end{figure}

% We also have lists:

% \begin{enumerate}
%   \item Static Analysis~\circled{3} examines program artifacts or
%     their source code without executing them~\cite{wichmann1995industrial}, while
%  \item Dynamic Analysis~\circled{4} relies on information gathered from their
%    execution~\cite{cornelissen2009systematic}.
% \end{enumerate}

% Or boxes:

% \begin{framed}
% This thesis is concerned with the empirical assessment of the state of the art of how developers
% drive software development with the help of feedback loops.
% \end{framed}

% Or code:
% \begin{lstlisting}[caption={\textsc{TrinityCore}},label={lst:e1}]
%  x += other.x;
%  y += other.y;
%  z += other.y;
% \end{lstlisting}




% Long: \acrlong{fdd}

% Short: \acrshort{fdd}

% Full: \acrfull{fdd}
